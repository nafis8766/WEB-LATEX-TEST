\section{Summary of Findings}
In our study, we point out the revolutionary potential of AI in the legal sphere, specifically, in the sphere of increasing efficiency of such tasks as research and document writing. But also, we identified that the existing Large Language Models (LLM) frequently fail to deal with legal intricacies and can generate misleading counterfeits, or false information, which hamper reliability. This was mitigated by the use of the Retrieval-Augmented Generation (RAG), which essentially meant minimized hallucinations by putting the answers in the perspective of real-life and current knowledge of the law. Although RAG tends to work reasonably well, specific adaptation to complicated multilingual legal environments is necessary since general RAG systems, in this case, can lack required domain knowledge. Our report was able to highlight explicitly the peculiarities of the landscape of the Bangladeshi legal system including the lack of digital sources, uneven access to the internet, and the denominal character of legal documents. To address these, we stress that it requires an intelligent, offline friendly legal AI system with techniques such as Context Awareness Gate (CAG) and  Bangla Natural Language Processing (NLP) that have the ability to negate unnecessary information. More than that, to support the deployment of mobile devices that are prevalent and have resource constraints, model compression algorithms, such as Knowledge Distillation (KD) and Quantization, are needed to decrease the model size and energy cost without compromising accuracy. Lastly, our findings support the idea that, despite the progress being made in multilingual LLMs, building specific Bengali models is essential to resolve the issue, such as the limited availability of data and ineffective text-processors, and make the legal AI solutions practical and effective in Bangladesh.

\section{Contributions to the Field}
Our work contributes both to the advancement of Natural Language Processing (NLP) and the broader societal application of AI technologies. Throughout our research we aimed to address the challenge of making large language models more efficient and accessible, particularly for use in specialized and sensitive domains like legal systems. We tried to show how a large model can be distilled to significantly reduce its size and resource requirement and still preserve its relevancy with the aid of Retrieval Augmented Generation (RAG). This is especially relevant for environments where access to powerful hardware or constant internet connectivity is limited. Additionally, aside from Bangladesh's legal domain, the work contributes to exploring the capabilities of domain-specific data and retrieval strategies can be used to work on edge devices in a decentralized manner preserving user privacy in a local language that remains underrepresented in mainstream research. 
\section{Recommendations for Future Work}
Few existing efforts address the combined challenges of making a small, reliable, interpretable and useful model that can be used offline in a local language on edge devices. Firstly, many legal systems suffer from fragmented, poorly digitized and unstructured legal documents so creating better datasets is essential. Secondly, this research focuses on offline deployment, however the next step should be to investigate how such systems can update securely and ethically preserving user privacy whenever connectivity is available. Thirdly, there is a lack of meaningful benchmark standards in legal NLP. Existing metrics like BLEU or ROUGE are not sufficient for tasks requiring legal precision or domain understanding so a domain specific benchmark involving legal professionals will be crucial. Lastly, these techniques used here should not be limited to law, the same principles can be applied to fields like healthcare, education and public policy which demands access to accurate, interpretable information with low computational requirements.